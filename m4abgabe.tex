\documentclass{sigchi}

% Use this command to override the default ACM copyright statement
% (e.g. for preprints).  Consult the conference website for the
% camera-ready copyright statement.

%% EXAMPLE BEGIN -- HOW TO OVERRIDE THE DEFAULT COPYRIGHT STRIP -- (July 22, 2013 - Paul Baumann)
% \toappear{Permission to make digital or hard copies of all or part of this work for personal or classroom use is      granted without fee provided that copies are not made or distributed for profit or commercial advantage and that copies bear this notice and the full citation on the first page. Copyrights for components of this work owned by others than ACM must be honored. Abstracting with credit is permitted. To copy otherwise, or republish, to post on servers or to redistribute to lists, requires prior specific permission and/or a fee. Request permissions from permissions@acm.org. \\
% {\emph{CHI'14}}, April 26--May 1, 2014, Toronto, Canada. \\
% Copyright \copyright~2014 ACM ISBN/14/04...\$15.00. \\
% DOI string from ACM form confirmation}
%% EXAMPLE END -- HOW TO OVERRIDE THE DEFAULT COPYRIGHT STRIP -- (July 22, 2013 - Paul Baumann)

% Arabic page numbers for submission.  Remove this line to eliminate
% page numbers for the camera ready copy
% \pagenumbering{arabic}

% Load basic packages
\usepackage{balance}  % to better equalize the last page
\usepackage{graphics} % for EPS, load graphicx instead 
\usepackage[T1]{fontenc}
\usepackage{txfonts}
\usepackage{mathptmx}
\usepackage[pdftex]{hyperref}
\usepackage{color}
\usepackage{booktabs}
\usepackage{textcomp}
% Some optional stuff you might like/need.
\usepackage{microtype} % Improved Tracking and Kerning
% \usepackage[all]{hypcap}  % Fixes bug in hyperref caption linking
\usepackage{ccicons}  % Cite your images correctly!
% \usepackage[utf8]{inputenc} % for a UTF8 editor only

% If you want to use todo notes, marginpars etc. during creation of your draft document, you
% have to enable the "chi_draft" option for the document class. To do this, change the very first
% line to: "\documentclass[chi_draft]{sigchi}". You can then place todo notes by using the "\todo{...}"
% command. Make sure to disable the draft option again before submitting your final document.
\usepackage{todonotes}

% Paper metadata (use plain text, for PDF inclusion and later
% re-using, if desired).  Use \emtpyauthor when submitting for review
% so you remain anonymous.
\def\plaintitle{Mobiles Tagebuch 2 - Abschlussbericht Nutzerstudie}
\def\plainauthor{David Fuchssteiner, Florian Pichlmann, Jakob Rathmair,
  Stefan Wallner}
\def\emptyauthor{}
\def\plainkeywords{HCI; Human-Computer-Interaction; Mobile App; Tagebuch; Android; Java}
\def\plaingeneralterms{Documentation, Standardization}

% llt: Define a global style for URLs, rather that the default one
\makeatletter
\def\url@leostyle{%
  \@ifundefined{selectfont}{
    \def\UrlFont{\sf}
  }{
    \def\UrlFont{\small\bf\ttfamily}
  }}
\makeatother
\urlstyle{leo}

% To make various LaTeX processors do the right thing with page size.
\def\pprw{8.5in}
\def\pprh{11in}
\special{papersize=\pprw,\pprh}
\setlength{\paperwidth}{\pprw}
\setlength{\paperheight}{\pprh}
\setlength{\pdfpagewidth}{\pprw}
\setlength{\pdfpageheight}{\pprh}

% Make sure hyperref comes last of your loaded packages, to give it a
% fighting chance of not being over-written, since its job is to
% redefine many LaTeX commands.
\definecolor{linkColor}{RGB}{6,125,233}
\hypersetup{%
  pdftitle={\plaintitle},
% Use \plainauthor for final version.
  pdfauthor={\plainauthor},
%  pdfauthor={\emptyauthor},
  pdfkeywords={\plainkeywords},
  bookmarksnumbered,
  pdfstartview={FitH},
  colorlinks,
  citecolor=black,
  filecolor=black,
  linkcolor=black,
  urlcolor=linkColor,
  breaklinks=true,
}

% create a shortcut to typeset table headings
% \newcommand\tabhead[1]{\small\textbf{#1}}

% End of preamble. Here it comes the document.
\begin{document}

\title{\plaintitle}

\numberofauthors{4}
\author{%
  \alignauthor{David Fuchssteiner\\
    \affaddr{for Submission}\\
    \affaddr{Vienna, Austrai}\\
    \email{e-mail address}}\\
  \alignauthor{Florian Pichlmann\\
    \affaddr{for Submission}\\
    \affaddr{Vienna, Austria}\\
    \email{florian.pichlmann@me.com}}\\
  \alignauthor{Jakob Rathmair\\
    \affaddr{for Submission}\\
    \affaddr{Vienna, Austria}\\
    \email{e-mail address}}\\
  \alignauthor{Stefan Wallner\\
    \affaddr{for Submission}\\
    \affaddr{Vienna, Austria}\\
    \email{e-mail address}}\\
}

\maketitle

\begin{abstract}
  UPDATED---\today. 
\end{abstract}

\category{H.5.m.}{Information Interfaces and Presentation
  (e.g. HCI)}{Miscellaneous} \category{See
  \url{http://acm.org/about/class/1998/} for the full list of ACM
  classifiers. This section is required.}{}{}

\keywords{\plainkeywords}

\section{Introduction}

\section{Motivation}
Background information about the problem, tasks, and users
\section{Related Work}
Other solutions to this problem or similar problems
Any previous work that you incorporated into your solution?
\section{Design}
Description of the design of your final App
Reasons for your design choices, e.g. what you learned from the earlier prototyping steps
\section{Implementation}
Brief(!) description of how the App was implemented (toolkits, languages, platforms)
Any serious implementation challenges you encountered and how you handled them
\section{Evaluation}
\subsection{How did we evaluate?}
Briefly describe the 3-4 usability tasks + the questionnaire;
Describe how you conducted the study (number of participantes, sequence of steps in the study, etc.)
\subsection{Results of our evaluation}
Explain what you learned about your App: strengths and weaknesses
Explain implications for design that you derived from the findings, and how you further improved the App respectively
\section{Reflection}
Give a clear separation of tasks between the group members for the entire project: you have to detail who did what for this milestone.
What were important lessons you learned from this course
\section{Conclusions and future work}
Summarize the main strengths and weaknesses of your approach and implementation
Give an outlook into intersting next steps that could be done building up on your work
\section{References}

\cite{CHINOSAUR:venue}

% REFERENCES FORMAT
% References must be the same font size as other body text.
\bibliographystyle{SIGCHI-Reference-Format}
\bibliography{biblio}

\end{document}

%%% Local Variables:
%%% mode: latex
%%% TeX-master: t
%%% End:
